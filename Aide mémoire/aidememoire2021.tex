\documentclass[12pt, a4paper]{book}
\usepackage[utf8]{inputenc}
\usepackage{amssymb}
\usepackage{amsmath}
\usepackage{amsthm}
\usepackage{mathtools}
\newtheorem*{definition}{Définition}
\newtheorem*{remarque}{Remarque}
\newtheorem*{theoreme}{Théorème}
\newtheorem*{proposition}{Proposition}
\newtheorem*{preuve}{Preuve}
\newtheorem*{observation}{Observation}
\newtheorem*{exemple}{Exemple}
\renewcommand{\chaptername}{Chapitre}
\let\oldemptyset\emptyset
\let\emptyset\varnothing
\setlength{\tabcolsep}{25pt}
\renewcommand{\arraystretch}{2.3}
\def\mathLarge#1{\mbox{\LARGE $#1$}}

\title{Aide mémoire 2021}
\date{\today}
\author{Léo Bernard}

\begin{document}
\maketitle
\tableofcontents

\chapter{Analyse}
\section{Notions de bases}
\subsection{Introduction}
On désigne par $\emptyset$ \textbf{l'ensemble vide. $\mathbb{N}$ l'ensemble des entiers naturels, $\mathbb{Z}$ l'anneau des entiers relatifs, $\mathbb{Q}$ le corps des nombres rationnels, et $\mathbb{R}$ le corps des nombres réels.} $\mathbb{R} \setminus \mathbb{Q}$ étant l'ensemble des \textbf{Nombres irrationnels.} On a donc les inclusions suivantes :\\
\begin{center}
    $\emptyset \subset \mathbb{N} \subset \mathbb{Z} \subset \mathbb{Q} \subset \mathbb{R}$
\end{center}
Par définition, on ajoute ${}^*$ pour signifier que le zéro est non compris dans l'ensemble, ${}_+$ pour signifier que l'ensemble ne contient que des positifs, ${}_-$ pour signifier que l'ensemble ne contient que des négatifs.
\subsection{Intervalle}
\begin{definition}
    Un sous ensemble $I \not = \emptyset$ de $\mathbb{R}$ est appelé un
    \textbf{intervalle} si pour tout couple $(a,b) \in I \times I$ vérifiant
     $a \leq b$, la relation $a \leq x \leq b$ implique $x \in I $.
\end{definition}
Il en découle une suite de notations :
\newpage
\subsubsection{Intervalles bornés}
Intervalle ouvert : $]a;b[ \ = ${$x \in \mathbb{R}:a < x < b$}\\
Intervalle fermé : $[a;b]\ = ${$x \in \mathbb{R}:a \leq x \leq b$}\\
Intervalle semi-ouvert à gauche : $]a;b]\ = ${$x \in \mathbb{R}:a < x \leq b$}\\
Intervalle semi-ouvert à droite : $[a;b[ \ = ${$x \in \mathbb{R}:a \leq x < b$}\\
\subsubsection{Intervalles non bornés}
Intervalle ouvert : $]-\infty;a[ \ = ${$x \in \mathbb{R}:x<a$}\\
Intervalle ouvert : $]a ;+\infty[\ = ${$x \in \mathbb{R}:x>a$}\\
\newline \\
Intervalle fermé : $]-\infty;a] \ = ${$x \in \mathbb{R}:x\leq a$}\\
Intervalle fermé :  $[a ;+\infty[ \ = ${$x \in \mathbb{R}:x \geq a$}\\

\subsection{Valeur absolue}
\begin{definition}
    A tout nombre réel $x$, on peut associer le nombre réel positif ou nul défini par :
    \begin{center}
        $$\lvert x \rvert =
        \begin{cases}
            x \text{ si } x > 0\\
            -x \text{ si } x \leq 0
        \end{cases}
        $$
    \end{center}
$\lvert x \rvert$ est appelé la \textbf{valeur absolue} de $x$.
\end{definition}
\subsection{Partie entière}
\begin{definition}
    A tout nombre réel $x$, on peut associer un unique entier relatif [$x$] tel que :
    \begin{center}
        $[x] \leq x < [x]+1$
    \end{center}
    $[x]$ est appelé la \textbf{partie entière} de $x$, soit le plus grand entier relatif inférieur ou égal à x.
\end{definition}
\newpage
\section{Fonction réelle}
Soit $A$ et $B$ deux sous ensembles de $\mathbb{R}$. On appelle fonction réelle une relation qui lie un élément $x$ de $A$ à un élément y ($f(x)$, la valeur de $f$ en $x$) dans $B$.

\begin{remarque}
    On appelle $A$ l'ensemble de départ et $B$ l'ensemble d'arrivée.
\end{remarque}
\begin{remarque}
    $x$ est aussi appelé la préimage de $y$ par $f$.
\end{remarque}
\begin{remarque}
L'ensemble des valeurs de $f$ est noté $Im(f)$.
\end{remarque}
\begin{remarque}
    Deux fonctions $f(x)$ et $g(x)$ sont dites égales si elles ont les mêmes ensembles d'arrivée et de départ, et si $f(x) = g(x)\ \ \forall x \in A$.\\
    On note alors $f = g$.\\
\end{remarque}
\subsection{Représentation graphique}

    On représente une fonction en dessinant l'ensemble des points de coordonnées $(a;f(a))$. Ce dessin est appelé \textbf{graphe de $f$}.

\begin{remarque}
    On appelle le nombre $a$ \textbf{zéro} de $f$ si $f(a)=0$. son ensemble correspond à l'ensemble des points ou le graphe de $f$ intersecte $O_x$
\end{remarque}
\newpage
\subsection{Parité d'une fonction}

    Si $f(-x) = f(x)$ pour tout x de l'ensemble de définiton de $f$, on dit que $f$ est une \textbf{fonction paire}. \\
    Si $f(-x) = -f(x)$ pour tout x de l'ensemble de définiton de $f$, on dit que $f$ est une \textbf{fonction impaire}. 

\begin{remarque}
    Le graphe d'une fontion paire est symétrique à l'axe $O_y$, et \\
    Le graphe d'une fontion impaire est symétrique à l'origine.\\
\end{remarque}

\subsection{Périodicité d'une fonction}
    Une fonction est dite de \textbf{période p} si il existe un nombre $p > 0$ tel que $f(x+kp) = f(x) \ \ \forall k \in \mathbb{Z}$\\

    \begin{remarque}
        Le graphe d'une fonction périodique est un motif qui se répète indéfiniment par translation horizontale (d'amplitude $p$).
    \end{remarque}

\subsection{Croissance et décroissance d'une fonction}

Pour tout $ x_1, x_2\ \ \in I$ on dit que :
\begin{itemize}
    \item  Une fonction $f$ est \textbf{croissante} sur un intervalle $I$ si :
    \begin{center}
        $x_1 < x_2 \Rightarrow f(x_1) \leq f(x_2)$
    \end{center}
    

    \item  Une fonction $f$ est \textbf{strictement croissante} sur un intervalle $I$ si :
    \begin{center}
        $x_1 < x_2 \Rightarrow f(x_1) < f(x_2)$
    \end{center}
   

    \item  Une fonction $f$ est \textbf{décroissante} sur un intervalle $I$ si :
    \begin{center}
        $x_1 < x_2 \Rightarrow f(x_1) \geq f(x_2)$
    \end{center}
   

    \item  Une fonction $f$ est \textbf{strictement décroissante} sur un intervalle $I$ si :
    \begin{center}
        $x_1 < x_2 \Rightarrow f(x_1) > f(x_2)$
    \end{center}
    
\end{itemize}
\newpage

\subsection{Maximum et minimum d'une fonction}

Soit $f : A \rightarrow \mathbb{R}$ une fonction réelle
\begin{itemize}
    \item  $f(a)$ est un \textbf{maximum local} de $f$ si il existe un intervalle ouvert $I$ contenant $a$ tel que :
    \begin{center}
        $\forall x \in I \cap A$ : $f(x) \leq f(a)$
    \end{center} 
    On dit aussi que $f$ admet un maximum en $a$.

    \item  $f(a)$ est un \textbf{minimum local} de $f$ si il existe un intervalle ouvert $I$ contenant $b$ tel que :
    \begin{center}
        $\forall x \in I \cap A$ : $f(x) \geq f(b)$
    \end{center} 
    On dit aussi que $f$ admet un minimum en $b$.

    \item  $f(a)$ est un \textbf{maximum absolu} de $f$ si :
    \begin{center}
        $\forall x \in A$ : $f(x) \leq f(a)$
    \end{center} 

    \item  $f(a)$ est un \textbf{minimum absolu} de $f$ si :
    \begin{center}
        $\forall x \in A$ : $f(x) \geq f(a)$
    \end{center} 
    
    
\end{itemize}
\begin{remarque}
    Le nom \textbf{extremum} peut être aussi utilisé à la place de maximum ou minimum.
\end{remarque}
\newpage

\subsection{Opérations sur les fonctions}
Soit $f : A \rightarrow \mathbb{R}$ et $f : B \rightarrow \mathbb{R}$ deux fonctions réelles

\begin{itemize}
    \item  La \textbf{somme} des fonctions $f$ et $g$ est une nouvelle fonction notée $f+g : A \cap B \rightarrow \mathbb{R}$ définie par :
    \begin{center}
        $(f + g)(x) = f(x)+g(x)$
    \end{center} 
   
    \item  La \textbf{différence} des fonctions $f$ et $g$ est une nouvelle fonction notée $f-g : A \cap B \rightarrow \mathbb{R}$ définie par :
    \begin{center}
        $(f - g)(x) = f(x)-g(x)$
    \end{center} 

    \item  Le \textbf{produit} de la fonction $f$ par un nombre réel $c$ est une nouvelle fonction notée $c*f : A \rightarrow \mathbb{R}$ définie par :
    \begin{center}
        $(c*f)(x) = fc*(x)$
    \end{center} 
    \item  Le \textbf{produit} des fonctions $f$ et $g$ est une nouvelle fonction notée $f*g : A \cap B \rightarrow \mathbb{R}$ définie par :
    \begin{center}
        $(f * g)(x) = f(x)*g(x)$
    \end{center} 

    \item  Le \textbf{quotient} des fonctions $f$ et $g$ est une nouvelle fonction notée $\frac{f}{g} : A \cap B \cap {x | g(x) \not = 0} \rightarrow \mathbb{R}$ définie par :
    \begin{center}
        $(\frac{f}{g})(x) = \frac{f(x)}{g(x)}$
    \end{center} 

    \item  La \textbf{composée} des fonctions $f$ et $g$ est une nouvelle fonction notée $g \circ f : {x|x \in A \ et \ f(x) \in B} \rightarrow \mathbb{R}$ définie par :
    \begin{center}
        $(g \circ f)(x) = g(f(x))$
    \end{center} 
\end{itemize}
\newpage
\subsection{Injection, surjection, bijection}
Soit une fonction $f \ A \rightarrow B$.
\begin{itemize}
    \item $f$ est dite \textbf{surjective} si tout élément $y$ de $B$ est l'image par $f$ d'au minimum un élément $x$ de $A$ (au \underline{minimum} une précédence pour chaque objet de $B$).

    \item $f$ est dite \textbf{injective} si tout élément $y$ de $B$ est l'image par $f$ d'au maximum un élément $x$ de $A$ (au \underline{maximum} une précédence pour chaque objet de $B$).

    \item $f$ est dite \textbf{bijective} si elle est \underline{à la fois injective et surjective}. Ainsi, chaque élément $y$ de $B$ est l'image par $f$ d'un \underline{unique} élément $x$ de $A$.

\end{itemize}
\subsection{Fonction réciproque}
Soit une fonction $f \ A \rightarrow B$ bijective.\\
On appelle \textbf{réciproque} de $f$ notée $^rf$ ou $f^{-1}$ la fonction $f^{-1} : B \rightarrow A$ définie par :
\begin{center}
    $y=f(x) \Leftrightarrow x = f^{-1}(y)$
\end{center}
Chaque fonction $f$ bijective peut donc avoir une fonction réciproque $f^{-1}$ tel que :
\begin{center}
    $(f^{-1} \circ f)(x) = x \ \forall x \in A$\\
    $(f \circ f^{-1})(y) = y \ \forall y \in B$
\end{center}
\newpage
\section{Limites}
\subsection{Limite : définition}
Soit $f$ une fonction définie sur un intervalle ouvert contenant $a$ sauf eventuellement en $a$.\\
Le nombre $L$ est \textbf{limite de $f$ en $a$ si $f(x)$} est arbitrairement proche de $L$ dès que $x$ tend vers $a$, avec $x \not = a$. 
On note : \begin{center}
    $\lim_{x \to a} f(x) = L$
\end{center}
On dit que $f(x)$ tend vers $L$ quand $x$ tend vers $a$.\\
\begin{remarque}
    On peut aussi utiliser les limites sur des suites plutôt que sur des fonctions.
\end{remarque}
\subsection{Limite à droite, limite à gauche}
Soit $f$ une fonction définie sur un intervalle de la forme $]a;d[$. Le nombre L est \textbf{limite à droite de f en a} si  $\lim_{x \to a_+} f(x) = L$\\
Soit $f$ une fonction définie sur un intervalle de la forme $]g;a[$. Le nombre L est \textbf{limite à gauche de f en a} si  $\lim_{x \to a_-} f(x) = L$\\
\subsection{Propriétés des limites}
Soit $f$ et $g$ deux fonctions admettant une limitent en $a$ et soit $\lambda \in \mathbb{R}$
\begin{center}
    \begin{tabular}{ |c|}
        \hline
        $\lim_{x \to a} [f(x)+g(x)] =\lim_{x \to a}f(x) + \lim_{x \to a}g(x) $\\
        $\lim_{x \to a} [f(x)-g(x)] =\lim_{x \to a}f(x) - \lim_{x \to a}g(x) $\\
        $\lim_{x \to a} [\lambda f(x)] =\lambda \lim_{x \to a}f(x)$\\
        $\lim_{x \to a} [f(x)*g(x)] =\lim_{x \to a}f(x) * \lim_{x \to a}g(x) $\\
        $\lim_{x \to a} [\frac{f(x)}{g(x)}] =\frac{\lim_{x \to a}f(x)}{\lim_{x \to a}g(x)}$ si $\lim_{x \to a}g(x) \not = 0$\\
        \hline
    \end{tabular}
\end{center}
\newpage
\subsection{Théorème des deux gendarmes}
\begin{center}
    \begin{tabular}{ |c|}
        \hline
        Soit $f$, $g$ et $h$ trois fonctions définies sur un intervalle ouvert $I$ contenant $a$,\\
        sauf éventuellement en $a$.\\
        Si $f(x) \leq h(x) \leq g(x) \forall x \in I \not \;$ \{a\} et si
        $\lim_{x \to a}f(x) =\lim_{x \to a}g(x) = L$\\
         alors $\lim_{x \to a}h(x) = L$\\
        \hline
    \end{tabular}
\end{center}
\subsection{Critère de d'Alembert}

\subsection{Critère de Cauchy}

\subsection{Continuité}
Une fonction $f$ est \textbf{continue} en $a$ si elle est définie sur une intervalle ouvert contenant $a$ et si $\lim_{x \to a}f(x) = f(a)$\\
Une fonction $f$ est \textbf{continue} en $a$ si elle est continue en tout point de l'intervalle $I$.\\
Une fonction $f$ est \textbf{continue sur un intervalle fermé} $[a;b]$ si elle est\\
continue en tout point de l'intervalle et si \\
$\lim_{x \to a_+}f(x) = f(a)$ et $\lim_{x \to b_-}f(x) = f(b)$.
\subsection{Limites de fonctions composées}
Soit $f$ et $g$ deux fonctions.
\begin{center}
    \begin{tabular}{|c|}
        \hline
        Si $\lim_{x \to a}f(x) = L$ et si de plus $g$ est continue en $L$, alors\\
        $\lim_{x \to a}g(f(x)) = g(\lim_{x \to a}f(x)) = g(L) $\\
        \hline
   \end{tabular}
   \newline \\
   \begin{tabular}{|c|}
    \hline
    Si $\lim_{x \to a}f(x) = L$ et si de plus $f(x) \not = L$ sur un intervalle ouvert\\
    contenant $a$, sauf éventuellement $a$, alors : \\
    $\lim_{x \to a}g(f(x)) = \lim_{t \to L}g(t)$\\
    \hline
\end{tabular}
\end{center}
\newpage
\subsection{Propriétés des fonctions continues}
\textbf{Continuité de la réciproque}
\begin{center}
    \begin{tabular}{|c|}
        \hline
        Soit $I$ un intervalle et $f:I \rightarrow J$ une fonction bbijective et continue.\\
        Alors la réciproque ${}^rf$ est continue sur l'intervalle $J$.\\
        \hline
    \end{tabular}
\end{center}
\textbf{Théorème de Bolzanno}
\begin{center}
    \begin{tabular}{|c|}
        \hline
        Si $f$ est continue sur l'intervalle $[a;b]$ et si $f(a)$ et $f(b)$ sont\\
        de signes différents, alors la fonction $f$ admet au moins un zéro dans\\
        $[a;b]$\\
        \hline
    \end{tabular}
\end{center}
\textbf{Théorème de la valeur intermédiaire}
\begin{center}
    \begin{tabular}{|c|}
        \hline
        Si $f$ est continue sur l'intervalle $[a;b]$, alors pour tout\\
        nombre $\gamma$ compris entre $f(a)$ et $f(b)$, il existe\\
        $c \in [a;b]$ tel que $f(c) = \gamma$\\
        \hline
    \end{tabular}
\end{center}
\textbf{Théorème de Bolzanno-Weierstrass}
\begin{center}
    \begin{tabular}{|c|}
        \hline
       L'image d'un intervalle fermé borné par une fonction continue est\\ 
       un intervalle fermé borné\\
        \hline
    \end{tabular}
\end{center}
\textbf{Corollaire}
\begin{center}
    \begin{tabular}{|c|}
        \hline
       Une fonction continue sur un intervalle fermé $[a;b]$ admet un\\
       maximum absolu et un minimum absolu sur cet intervalle.\\
        \hline
    \end{tabular}
\end{center}
\subsection{Limites infinies}
Soit $f$ une fonction définie sur un intervalle ouvert contenant $a$, sauf éventuellement en $a$.\\
\\
on écrit $\lim_{x \to a}f(x) = + \infty$ si $f(x)$ est arbitrairement grand quand x tend vers a, avec $x \not = a$.\\
on écrit $\lim_{x \to a}f(x) = - \infty$ si $\lim_{x \to a}(-f(x)) = + \infty$
\subsection{Propriétés des limites infinies}
\begin{center}
    \begin{tabular}{ |c|}
        \hline
        $\lim_{x \to a} f(x) = L$ et $\lim_{x \to a} g(x) = + \infty \Rightarrow \lim_{x \to a} [f(x)+g(x)] = + \infty $\\
        $\lim_{x \to a} f(x) = L < 0$ et $\lim_{x \to a} g(x) = + \infty \Rightarrow \lim_{x \to a} [f(x)*g(x)] = - \infty $\\
        $\lim_{x \to a} f(x) = L \not = 0$ et $\lim_{x \to a} g(x) = 0 \Rightarrow \lim_{x \to a} \lvert \mathLarge{\frac{f(x)}{g(x)}} \rvert = + \infty $\\
        $\lim_{x \to a} f(x) = L \not = 0$ et $\lim_{x \to a} g(x) = \pm \infty \Rightarrow \lim_{x \to a} \mathLarge{\frac{f(x)}{g(x)}}  = 0 $\\
        \hline
    \end{tabular}
\end{center}
\begin{remarque}
    $\frac{0}{0}$, $\frac{\infty}{\infty}$, $0*\infty$ et $\infty-\infty$ sont des formes dites indéterminées.
\end{remarque}
\subsection{Limites à l'infini}
Soit $f$ une fonction définie sur un intervalle de la forme $[a;+\infty[$.\\
On écrit $\lim_{x \to \infty} f(x) = L$ si $f(x)$ est arbitrairement proche de $L$ quand\\
$x$ est suffisamment grand ou de manière équivalente, si $\lim_{t \to 0_+} f(\frac{1}{t}) = L$.\\
Soit $f$ une fonction définie sur un intervalle de la forme $]- \infty; a]$. On écrit $\lim_{x \to -\infty} f(x) = L$ si $\lim_{x \to +\infty} f(-x) = L$
\newpage
\subsection{Asymptotes}
\begin{definition}
    La droite d'équation $x=a$ est une \textbf{asymptote verticale} de la fonction $f$ si\\
    $\lim_{x \to a_+} \lvert f(x) \rvert = + \infty$ ou si  $\lim_{x \to a_-} \lvert f(x) \rvert= +\infty$
\end{definition}
\begin{definition}
    La droite d'équation $y=h_1$ est une \textbf{asymptote horizontale} de la fonction $f$ vers $+\infty$ si $\lim_{x \to + \infty}  f(x)  = h_1$
\end{definition}
\begin{definition}
    La droite d'équation $y=h_2$ est une \textbf{asymptote horizontale} de la fonction $f$ vers $-\infty$ si $\lim_{x \to - \infty}  f(x)  = h_2$
\end{definition}
\begin{definition}
    La droite d'équation $y=h_1$ est une \textbf{asymptote horizontale} de la fonction $f$ vers $+\infty$ si $\lim_{x \to + \infty}  f(x)  = h_1$
\end{definition}
\begin{definition}
    La droite d'équation $y=mx+h$ est une \textbf{asymptote oblique} de la fonction $f$ vers $+\infty$ si \\$f(x) = mx + h + \delta(x)$ avec $\lim_{x \to  +\infty}\delta(x)  = 0$
\end{definition}
\begin{definition}
    La droite d'équation $y=mx+h$ est une \textbf{asymptote oblique} de la fonction $f$ vers $-\infty$ si \\$f(x) = mx + h + \delta(x)$ avec $\lim_{x \to  -\infty}\delta(x)  = 0$
\end{definition}
\subsection{Astuces de calcul}
\subsubsection{Division euclidienne}
Soit $f(x)$, $g(x)$ deux fonctions rationnelles et $L =\lim_{x \to + b}  \frac{f(x)}{g(x)}  = "\frac{0}{0}"$,\\
avec b $\in \mathbb{R}$\\
On remarque dans cette situation que $f(x)$ et $g(x)$ sont divisibles par $(x-b)$ (car quand $x = b$, $f(x)$ et $g(x)$ sont tous deux nuls.)\\
Ainsi, nous pouvons mettre en évidence $(x-b)$ dans $f(x)$ et $g(x)$ grâce a une divison euclidienne (un schéma de Horner peut s'avérer pratique).
\begin{remarque}
    Si le reste de la division euclidienne est de zéro, il peut être intéressant de remettre $(x-b)$ en évidence dans la nouvelle limite obtenue.
\end{remarque}
\subsubsection{Règle de Bernouilli-L'Hospital}
Se réferer à la définition de la règle de Bernouilli L'Hospital.\\
Rappel rapide :\\
\begin{center}
    $\lim_{x \to a} \frac{f'(x)}{g'(x)} = \lim_{x \to a} \frac{f(x)}{g(x)}  = L$
\end{center}
\textit{}{Avec $g$ et $g'$ non nuls, et $f$, $g$ dérivables.}
\subsubsection{Limites remarquables}
\begin{tabular}{ |c | c| } 
    \hline
Fonctions Trigonométriques :&Fonctions Logarithmiques :\\
\hline
    $\lim_{x \to 0} \frac{sin(x)}{x} = 1$& $\lim_{x \to 0} \frac{e^x-1}{x} = 1$\\
    $\lim_{x \to 0} \frac{1-cos(x)}{x^2} = \frac{1}{2}$&$\lim_{x \to 0} (1 + \frac{a}{x})^x = e^a$\\
    $\lim_{x \to 0} \frac{arcsin(x)}{x} = 1$&$\lim_{x \to 0} \frac{ln(1+ax)}{x} = a$\\
    $\lim_{x \to 0} \frac{tan(x)}{x} = 1$&$\lim_{x \to 0} (1+ax)^\frac{1}{x} = e^a$
\end{tabular}


\newpage
\section{Dérivées}
        \subsection{Tangeante (dérivée) en $x_0$}
            Soit deux points M et $M_0$, définis par : $M_0(x_0;f(x_0))$ et $M(x;f(x))$.\\
            Quand $x$ tend vers $x_0$, alors $M$ s'approche de $M_0$ et la droite $(M_0M)$
            tend vers une droite limite que l'on appelle $\mathbf{tangeante}$ à f(x) en $M_0$.
            Cette tangeante en $x_0$ est nommée dérivée de $f$ au point $x_0$, et sa pente est donnée par la limite :\\
            $$
            \boxed{
            f'(x_0) := m = \lim_{x \to x_0} \frac{f(x)-f(x_0)}{x-x_0}
            }
            $$
\subsection{Nombre dérivé à gauche, à droite}
    La notion de limite a gauche (resp. à droite) permet de définir le nombre dérivé à gauche (resp. droite) d'une fonction en un point.
    Ceci nous permet de déterminer si la fonction est $\textbf{dérivable en ce point}$ si les limites à gauche et à droite sont les mêmes.\\
    Ainsi :
    $$
    \boxed{
    f'(x_{0}) \text{ existe si : }= \lim_{x \to x_{0-}} \frac{f(x)-f(x_0)}{x-x_0} = \lim_{x \to x_{0+}} \frac{f(x)-f(x_0)}{x-x_0}
    }
    $$\\
    \textit{(notons ici que $x_{0-}$ et $x_{0+}$ représentent un nombre légèrement plus petit que x, et  resp. un nombre légèrement plus grand que x.)}
\subsection{Point anguleux, à tangeante verticale, de rebroussement}
\begin{itemize}

    \item Le graphe d'une fonction $f$ admet un $\textbf{point anguleux en a}$ si $f$ est continue en $a$ et si :\\
$$\boxed{
\lim_{x \to a_{-}}f'(x) \not = \lim_{x \to a_{+}}f'(x)
}
$$\\
\textit{(si la fonction $f$ est continue en $a$ mais non dérivable en $a$ alors $f$ admet un point anguleux en $a$.)}\\
\\
\item Le graphe d'une fonction $f$ admet une $\textbf{tangeante verticale en a}$ si $f$ est continue en $a$ et si :\\
$$\boxed{
\lim_{x \to a_{-} } \lvert f'(x) \rvert = +\infty
}
$$\\
\textit{ce point est un \textbf{point de rebroussement} si de plus la limite $\lim_{x \to a_{-}}f'(x) $ n'existe pas.}\\
\\
\end{itemize}
\subsection{fonction dérivée}
\begin{definition}
    Une fonction $f$ est \textbf{dérivable} sur une partie de $A$ sur $\mathbb{R}$ si elle est dérivable en tout points de $A$. On définit la fonction dérivée par :\\
    \begin{center}
        $f': A \to \mathbb{R}$\\
        $x \to f'(x)$
    \end{center}
\end{definition}
\pagebreak
$\textbf{Dérivées de fonction élémentaires}$
\begin{center}
    \begin{tabular}{ |c | c| } 
     \hline
     $f(x)$& $f'(x)$\\ 
     \hline
     $c$ & 0 \\ 
     $x$ & 1 \\ 
     $x^n$ \ \ \ $n\in \mathbb{N^*}$ & $n*x^{n-1}$  \\ 
     $e^u$  & $u' e^u$ \\ 
     $\frac{1}{x}$ & $\mathLarge{-\frac{1}{x^2}}$ \ \ \ $ x\not = 0$ \\ 
     $\sqrt{x}$ & $\mathLarge{\frac{1}{2\sqrt{x}}}$ \ \ \ $x>0$ \\ 
     $cos(x)$ & $-sin(x)$\\
     $\lvert x \rvert$ & $sgn(x)$\ \ \ $x \not = 0$\\ 

     \hline
    \end{tabular}
\end{center}
    $\textbf{Dérivées de fonction particulières}$
    \begin{center}
    \begin{tabular}{ |c |c | }
     \hline
     $f(x)$& $f'(x)$ \\ 
     \hline
     $x^q$ \ \ \ $q \in \mathbb{Q}$& $qx^{q-1}$ \\  
     $tan(x)$ &  $\mathLarge{\frac{1}{cos^2(x)}} = 1+tan^2(x)$\\ 
     $cot(x)$ &   $\mathLarge{\frac{-1}{sin^2(x)}} = -1-cot^2(x)$ \\ 
     $arcsin(x)$ &  $\mathLarge{\frac{1}{\sqrt{1-x^2}}}$   \\ 
     $arccos(x)$ & $\mathLarge{\frac{-1}{\sqrt{1-x^2}}}$  \\
     $arctan(x)$ & $\mathLarge{\frac{1}{1+x^2}}$   \\

     \hline
    \end{tabular}
    \end{center}
\subsection{Dérivée d'ordre supérieur}
\begin{definition}
    La \textbf{dérivée d'ordre n} de $f$ est la fonction n fois dérivée $f^{(n)}$ définie par $f^{(n)}(x) = (f^{(n-1)})'(x)$ 
\end{definition}
\subsection{Règle de Bernouilli-L'Hospital}
\begin{definition}
    Soient des fonctions $f$, $g$ telles que $f, g : ]a,b[ \rightarrow F$, \\
    dérivables telles que $g, g'$ ne s'annulent pas sur $]a,b[$. De plus, on suppose que :
    \begin{itemize}
        \item  $\lim_{x \to a} f(x) = \lim_{x \to a} g(x) = \alpha$ avec $\alpha= 0, - \infty$ ou $+\infty$;
        \item $\lim_{x \to a} \frac{f'(x)}{g'(x)} = L$, avec $L \in \mathbb{R} \cup {-\infty, +\infty}$
    \end{itemize}
Alors,
\begin{center}
    \boldmath{$\lim_{x \to a} \frac{f'(x)}{g'(x)} = L$}
\end{center}
\end{definition}
\begin{remarque}
    cette règle reste valable quand x tend vers b-, a+, -$\infty$ ou +$\infty$
\end{remarque}
\subsection{Propriétés utiles}
    \begin{tabular}{ |c|}
        \hline
        Toute fonction dérivable en $a$ est continue en $a$\\
        \hline
    \end{tabular}
    \newline \\
    \begin{tabular}{ |c|}
        \hline
        Si la fonction $f$ est dérivable en $a$ et admet un extremum en $a$, alors $f'(a) = 0$\\
        \hline
    \end{tabular}
    \newline \\
\textbf{Théorème de Rolle}\\
\newline \\
\begin{tabular}{ |c|}
    \hline
    Si $f$ est une fonction continue sur l'intervalle $[a;b]$, et dérivable sur\\
     l'intervalle $]a;b[$ et si $f(a) = f(b)$ alors il existe au moins\\
    un nombre c dans $]a;b[$ t.q. $f'(c)=0$\\
    \hline
\end{tabular}
\newline \\
\textit{(Il existe entre les points $A$ et $B$ de "même hauteur" un point ayant une tangeante horizontale.)}\\
\newline \\
\textbf{Théorème des accroisements finis (TAF)}\\
\newline \\
\begin{tabular}{ |c|}
    \hline
    Si $f$ est une fonction continue sur l'intervalle $[a;b]$, et dérivable \\
    sur l'intervalle $]a;b[$ alors il existe au moins un nombre c \\
    dans $]a;b[$ t.q. $f'(c)= \mathLarge{\frac{f(b)-f(a)}{b-a}}$\\
    \hline
\end{tabular}
\newline \\
\textit{(Il existe entre les points $A$ et $B$ un point ayant une tangeante parrallèle à la droite $AB$.)}
\newpage
\subsection{Règles de dérivation}
Soit $f$ et $g$ deux fonction dérivables en $a$. Soit $c \in \mathbb{R}$\\
\begin{center}
\begin{tabular}{ |c|}
    \hline
    $(f+g)'(a) = f'(a) + g'(a)$\\
    $(f-g)'(a) = f'(a) - g'(a)$\\
    $(c*f)'(a) = c*f'(a)$\\
    $(f*g)'(a) = f'(a)*g(x) + f(a)*g'(a)$\\
    $(\frac{f}{g})'(a) = \mathLarge{\frac{f'(a)*g(a) - f(a)g'(a)}{g^2(a)}}$\\
    \hline
\end{tabular}
\end{center}

Si $f$ est une fonction dérivable en $a$ et g une fonction dérivable en $f(a)$, alors $g \circ f$ est dérivable en $a$ et :\\
\begin{center}
    \begin{tabular}{ |c|}
        \hline
        $(g \circ f)'(a) = g'(f(a)) * f'(a)$\\
        \hline
    \end{tabular}
    \end{center}
\begin{exemple}
    On "dérive en boîtes" :\\
    \newline \\
    $\rightarrow sin^{2}(2x)' =$\\
    \newline \\
    \textcircled{1} Dériver le carré : $2sin(2x)$\\
    \textcircled{2} Dériver le sinus : $cos(2x)$\\
    \textcircled{3} Dériver 2x : $2$\\
    \textcircled{4} Multiplier chaque partie entre elles :$2sin(2x)*cos(2x)*2 = 4sin(2x)cos(2x)$\\
\end{exemple}
\newpage
\section{Intégrales}
\subsection{Introduction}
Soient a $<$ b deux éléments de $\mathbb{R}$. Le sous-ensemble fini et ordonné
\begin{center}
    $\sigma = \{x_0 = a, x_1,$ ... $,x_{n-1},x_n =b\}$ avec $a < x_1< ... < x_{n-1} < b $
\end{center}
est appelé une \textbf{subdivision} de l'intégrale [a;b].

\subsubsection{Somme de Riemann supérieure}
Soit $\sigma = \{x_0, x_1, ... ,x_n\}$ une subdivision de $[a,b]$ et $f:[a,b] \rightarrow \mathbb{R}$ une fonction continue.\\
On nomme \textbf{somme de Riemann Supérieure} le nombre :
\begin{center}
    $\bar{S}_{\sigma}(f)=  \sum_{k=1}^{n} f(t_k)(x_k-x_{k-1})$
\end{center}
avec  $f(t_k) = $sup$ \{f(t), t_k \in [x_{k - 1}, x_k] \}$

\subsubsection{Somme de Riemann inférieure}
Soit $\sigma = \{x_0, x_1, ... ,x_n\}$ une subdivision de $[a,b]$ et $f:[a,b] \rightarrow \mathbb{R}$ une fonction continue.\\
On nomme \textbf{somme de Riemann Inférieure} le nombre :
\begin{center}
    $\underbar{S}_{\sigma}(f)=  \sum_{k=1}^{n} f(t_k)(x_k-x_{k-1})$
\end{center}
avec  $f(t_k) = $inf$ \{f(t), t_k \in [x_{k - 1}, x_k] \}$

\subsubsection{Intégrale Bornée}
Soit $\underbar{S}(f)$, $\bar{S}(f)$, deux nombres réels tels que :\\
\begin{center}
    $\underbar{S}(f)=$ inf\{$\underbar{S}_{\sigma}(f) : \sigma$ subdivision de $[a,b]$\}\\
    et\\
    $\bar{S}(f)=$ inf\{$\bar{S}_{\sigma}(f) : \sigma$ subdivision de $[a,b]$\}
\end{center}
Soient a $<$ b deux éléments de $\mathbb{R}$ et $f:[a,b]\rightarrow \mathbb{R}$ une fonction continue. Par définition, le nombre réel $\underbar{S}(f)=\bar{S}(f)$ est
appelé l'\textbf{intégrale} de la fonction $f$ sur $[a,b]$ et on écrit :
\begin{center}
    $\underbar{S}_{\sigma}(f)=\bar{S}_{\sigma}(f) = \int_{a}^{b} f(x) dx $
\end{center}

\subsection{Primitives d'une fonction}
\begin{definition}
    Soit $f$ une fonction définie sur un intervalle $I$ (une partie de $\mathbb{R}$). Une fonction dérivable $F$ est une \textbf{primitive} de $f$ sur $I$ si\\
     $F'(x) = f(x) \forall x \in I$.
\end{definition}
On désigne généralement par $\int f(x)dx$ l'ensemble des primitives de $f$ sur $I$. On l'appelle \textbf{intégrale indéfinie} de f.\\
\textbf{Intégrer} une fonction $f$ sur un intervalle $I$ c'est chercher toutes les primiteives de $f$ sur $I$.\\
Si $F$ est primitive de $f$ sur $I$, alors toute primitive de $f$ est de la forme $\mathbf{F(x)+c}$, avec $c \in \mathbb{R}$. On convient d'écrire :\\
\begin{center}
    \begin{tabular}{ |c|}
        \hline
       $\int f(x)dx = F(x)+c, \ \ c \in \mathbb{R}$\\
        \hline
    \end{tabular}
\end{center}
\subsubsection{Théorème fondamental du calcul intégral}
Soient a $<$ b deux éléments de $\mathbb{R}$ et $f:[a,b]\rightarrow \mathbb{R}$ une fonction continue.
Alors, si $F:[a,b]\rightarrow \mathbb{R}$ est une primitive on écrit :
\begin{center}
    $F(b)-F(a)=\int_{a}^{b} f(x) dx$\\
    \ \\
    on utilise aussi la notation :\\
    \ \\
    $\int_{a}^{b} f(x) dx = F(x)|^b _a$
\end{center}
\subsection{Intégration par parties}
Soient $I$ un intervalle ouvert, $a,b \in I$ et $f,g:I \rightarrow \mathbb{R}$ deux fonctions telles que
$f,g$ soient dérivables sur $I$, et $f',g'$ sont continues sur $I$. Alors :\\
\begin{center}
    $\int_{a}^{b} f(x)g'(x) dx = f(x)g(x)|^b _a- \int_{a}^{b} f'(x)g(x) dx$\\
    \ \\
    ou plus généralement :\\
    \ \\
    $u*v = \int u*v' + \int u'* v$\\
    \ \\
    on peut le montrer en intégrant :\\
    \ \\
    $(u*v)'= u'*v + u*v'$
\end{center}
\subsection{Changement de variable}
Soit $I$ un intervalle réel, $\phi : [a,b] \rightarrow I$ une fonction dérivable avec la dérivée intégrable, et $f: I \rightarrow \mathbb{R}$ une fonction continue. Alors :
\ \\
\begin{center}
    $\int_{a}^{b} f(\phi(t))\phi'(t)dt = \int_{\phi(a)}^{\phi(b)} f(x) dx$\\
\end{center}
Par définition, la transformation
\begin{center}
    $x = \phi(t) $ avec $ t\in I$
\end{center}
est appelé un \textbf{changement de variable.}
\subsubsection{Changements de variables communs / recommandés}
à remplir
\newpage
\subsubsection{Primitives de fonctions élémentaires}
\begin{center}
    \begin{tabular}{|c|c|}
        \hline
       $f(x)$ & $\int f(x)dx$\\
        \hline
        $a$ & $ax+c \ \ \ c \in \mathbb{R}$\\
        $x^q \ \ q \in \mathbb{Q} \setminus \{-1\}$ & $\frac{x^{q+1}}{q+1}+c \ \ \ c \in \mathbb{R}$\\
        $\frac{1}{x}$ & ln $\lvert x \rvert + c \ \ \ c \in \mathbb{R}$\\
        $e^x$ & $e^x + c \ \ \ c \in \mathbb{R}$\\
        ln $x \ \ \ a \not = 1, a>0, x>0$& $x($ln $x-1)+c \ \ \ c \in \mathbb{R}$\\
        $cos(x)$ & $sin(x)+c \ \ \ c \in \mathbb{R}$\\
        $sin(x)$ & $-cos(x)+c \ \ \ c \in \mathbb{R}$\\
        \hline
    \end{tabular}
\end{center}
\subsubsection{Primitives de fonctions particulières}
\begin{center}
    \begin{tabular}{|c|c|}
        \hline
       $f(x)$ & $\int f(x)dx$\\
        \hline
        $\frac{1}{a^2+x^2} \ \ \ a \not = 0$& $\frac{1}{a}Arctg(\frac{x}{a})+c \ \ \ c \in \mathbb{R}$\\
        $a^x \ \ \ a \not = 1, a>0$& $\frac{a^x}{ln(a)}+c \ \ \ c \in \mathbb{R}$\\
        $log_ax \ \ \ a \not = 1, a>0, x>0$& $x(log_ax-log_ae)+c \ \ \ c \in \mathbb{R}$\\
    
        \hline
    \end{tabular}
\end{center}
\newpage
\subsection{Règles d'intégration}
Soit $f$ et $g$ deux fonctions admettant une primitive sur un intervalle $I$
\begin{center}
    \begin{tabular}{|c|}
        \hline
        $\int \lambda f(x)dx = \lambda \int f(x)dx \ \ \ \lambda \in \mathbb{R}$\\
        $\int (f(x)+g(x))dx = \int f(x)dx + \int g(x)dx $\\
        $\int (f(x)-g(x))dx = \int f(x)dx - \int g(x)dx $\\
        $\int g(f(x))*f'(x)dx = G(f(x)) + c \ \ \ c \in \mathbb{R}$\\
        Où $G$ est une primitive de $g$.\\
        \hline
    \end{tabular}
\end{center}
\newpage
\section{Applications des dérivées}


\end{document}